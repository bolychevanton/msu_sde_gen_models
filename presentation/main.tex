\documentclass[xcolor=dvipsnames]{beamer}

\usepackage{pgf}
\usepackage{etex}
\usepackage{tikz,pgfplots}
\usepackage[utf8]{inputenc}
\usepackage[russian]{babel}
\usepackage{xcolor}
\usepackage{graphicx}
\usepackage{MnSymbol}
\usepackage{comment}
\usepackage[utf8]{inputenc}
\usepackage{pdfpages}
\usepackage{listings}
\usepackage{color}
\usepackage{booktabs}
\usepackage{soul}
\usepackage[normalem]{ulem}
\usepackage{tcolorbox}
\usepackage{lipsum}
\usepackage{tikz}
\usepackage{listings}
\usepackage{color}
\usepackage{amsmath,amsfonts}
\usepackage{algorithmic}
\usepackage{algorithm}
\usepackage{array}
%%%%%%%%%%%%%%%%%%%%%%%%%%%%%%%%%%%%%%%%%%%%%%%%%

\usetheme{Antibes}
%\usetheme{Madrid}
%\usecolortheme[named=Maroon]{structure}
\usecolortheme{dolphin}
\usefonttheme{professionalfonts}
\useoutertheme{infolines}
\useinnertheme{circles}

\newtheorem*{bem}{Bemerkung}


\definecolor{dkgreen}{rgb}{0,0.6,0}
\definecolor{gray}{rgb}{0.5,0.5,0.5}
\definecolor{mauve}{rgb}{0.58,0,0.82}

\lstset{frame=tb,
  language=Java,
  aboveskip=2mm,
  belowskip=2mm,
  showstringspaces=false,
  columns=flexible,
  basicstyle={\small\ttfamily},
  numbers=none,
  numberstyle=\tiny\color{gray},
  keywordstyle=\color{blue},
  commentstyle=\color{dkgreen},
  stringstyle=\color{mauve},
  breaklines=true,
  breakatwhitespace=true,
  tabsize=2
}



\newtheorem{remark}{Remark}
% \newtheorem{definition}{Definition}
\newtheorem*{todo}{TODO}
\newtheorem{question}{Вопрос}
\newtheorem*{remark*}{Замечание}
% \newtheorem{theorem}{Theorem}
% \newtheorem{corollary}{Corollary}
\newtheorem*{corollary*}{Corollary}
% \newtheorem{lemma}{Lemma}
\newtheorem*{prop}{Предложение}
\newtheorem{proposition}{Утверждение}

\newcommand{\Id}{\operatorname{Id}}
\newcommand{\HK}[1]{\mathcal{HK}[#1]}
\newcommand{\Real}{\mathbb{R}}
\newcommand{\Exp}{\mathbb{E}}
% \newcommand{\Prob}{\mathbb{P}}
\newcommand{\eps}{\varepsilon}
\newcommand{\boldx}{\boldsymbol{x}}
\newcommand{\Boldx}{\mathbf{x}}
\newcommand{\boldzero}{\boldsymbol{0}}
\newcommand{\boldy}{\boldsymbol{y}}
\newcommand{\Boldy}{\mathbf{y}}
\newcommand{\boldz}{\boldsymbol{z}}
\newcommand{\bolda}{\boldsymbol{a}}
\newcommand{\boldv}{\boldsymbol{v}}
\newcommand{\boldb}{\boldsymbol{b}}
\newcommand{\boldw}{\boldsymbol{w}}
\newcommand{\boldu}{\boldsymbol{u}}
\newcommand{\boldxi}{\boldsymbol{\xi}}
\newcommand{\calP}{\mathcal{P}}
\newcommand{\boldeta}{\boldsymbol{\eta}}
\newcommand{\pinv}[1]{#1 ^ {\mathrm{g}}}
\newcommand{\Ind}[1]{\mathbf{1}_{\left\{#1\right\}}}
\newcommand{\Mat}[2]{\operatorname{Mat}_{#1 \times #2}(\Real)}
\newcommand{\Beps}{B_{\eps}}
\newcommand{\notBeps}{\overline{B}_{\eps}}
%%%%%%%%%%%%%%%%%%%%%%%%%%%%%%%%%%%%%%%%%%%%%%%%%

\title[%
    % Краткое название работы не используется в этой презентации!
    SDE for Generative Modeling 
]{%
    % Полное название работы отображается на титульной странице
    Stochastic Differential Equations for Generative Modeling 
}

\author[%
    % Имя и фамилия автора работы отображаются на каждом слайде в нижнем колонтитуле
    
]{%
    % Имя, отчество и фамилия автора работы отображаются на титульном слайде
    \small
    Anton Bolychev\inst{1,2},
    Georgiy Malaniya\inst{2},
    Oleg Shepelin\inst{2},
    Anastasiya Archangelskaya\inst{2}, 
    Nikolay Kalmykov\inst{2},
    Vadim Shirokinsky\inst{2}
    \\[0.2cm]
    The presentation was developed by 1st year MSU Phd Student\\
    \normalsize
    \textbf{Anton Bolychev\inst{1,2}}
}

\date[%
    May 2023
]{%
    May 2023
}

\institute[%
    % Краткое название организации не используется в этой презентации
    MSU, Skoltech]{%
    % Полное название организации и подразделения
    \inst{1}
    Moscow State University 
    Faculty of Mechanics and Mathematics \\
    \inst{2}
    Skolkovo Institute Of Science and Technology
}



\begin{document}
    \begin{frame}
        \titlepage
    \end{frame}
    \begin{frame}{Contribution}
        This work was developed as the final project for the \textbf{Evgeniy Burnaev}'s course  
        \textbf{Machine Learning} that was held in Skolkovo Intsitute of Science and Technology.
        Contribution of all the participants is as follows:
        \begin{enumerate}
            \item \textbf{Anton Bolychev} participated as the \textbf{Principal Developer} in the present work. The core
            research, the code, the repo structure was prepared and finalized by him. Moreover he is the only 
            author of the presentation.
            \item \textbf{Georgiy Malaniya} particapated as the \textbf{Team Lead}. He maintained all the management
            process of all the participants
            \item \textbf{Oleg Shepelin} and \textbf{Nikolay Kalmykov} participated as 
            the \textbf{Core Developers} and prepared the draft of main loop for Langevin Dynamics
            \item \textbf{Vadim Shirokinsky} was responsible for the calculating FIDs routine and took part in preparing
            the presentation for Skoltech.
            \item \textbf{Anastasiya Archangelskaya} finalized results and prepared the report with presentation for
            Skoltech
        \end{enumerate}
    \end{frame}
    \section{GAN as Energy-Based Model}
    \label{sec:energy-gan} 
    \begin{frame}{GAN}
        \begin{align*}
            L_D &= -\mathbb{E}_{x \sim p_{data}} [ \text{log} D(x)] - \mathbb{E}_{z \sim p_z} [\text{log}(1 - D(G(z))] \\    
            L_G &= -\mathbb{E}_{z \sim p_z} [\text{log} D(G(z))]
        \end{align*}
    \end{frame}
    
    \begin{frame}{GAN as Energy Based Model}
        Assume that Discriminator is suboptimal, i.e. $D = D^*$
        \begin{equation*}
            D(x) = \operatorname*{logit}(d(x)) = \frac{1}{1 + \exp(-d(x))} \approx \frac{p_d(x)}{p_d(x) + p_g(x)}  = \frac{1}{1 + p_g(x)/p_d(x)}
        \end{equation*}
        Thus,
        \begin{equation*}
                p_d^*(x) = p_g(x) e^{d(x)} / K = \exp\left( - (-\log p_g(x) - d(x))\right) / K
        \end{equation*}
        Energy Function. Boltzmann distribution
        \begin{equation*}
            p(z) = \exp(-E(z)) / K 
        \end{equation*}
        Thus, assuming that $x = G(z)$ one can obtain the following equation for Energy for GAN
        \begin{equation*}
            E(z) = -\log p_0(z) - d(G(z))
        \end{equation*}
    \end{frame}

    \begin{frame}{Langevin Dynamics}
        \begin{equation*}
            z_{i+1} = z_i - \epsilon/2 \nabla_z E(z) + \sqrt{\epsilon} n, n \sim N(0, I)
        \end{equation*}

        \begin{center}
         \begin{algorithmic}
            \STATE {\bfseries Input:} N$\in \mathbf{N}_+$, $\epsilon > 0$ 
            \STATE {\bfseries Output:} Latent code $z_N \sim p_t(z)$
            \STATE Sample $z_0 \sim p_0(z)$   
            \FOR{$i=1$ {\bfseries to} $N$}
            \STATE $n_i \sim N(0, 1)$
            \STATE $z_{i + 1} = z_i - \epsilon/2 \nabla_z E(z_i) + \sqrt{\epsilon} n_i$
            \ENDFOR
         \end{algorithmic}
        \end{center}
    \end{frame}
    \begin{frame}{Frechet Inception Distance}
        FID can be calculated according to the following formula
        \begin{equation*}
            d_F(\mu, \nu):=\left(\inf _{\gamma \in \Gamma(\mu, \nu)} \int_{\mathbb{R}^n \times \mathbb{R}^n}\|x-y\|^2 \mathrm{~d} \gamma(x, y)\right)^{1 / 2}
        \end{equation*}
    \end{frame}
    \begin{frame}{\hypertarget{frame:res-energy}{Results}} 
        \begin{center}
            If one apply Langevin Dynamics for pretrained DCGAN on CIFAR10 one can observe 
            that FID metrics decreases with increasing number of Langevin steps.
            \includegraphics[width=0.7\textwidth]{pics/FID_EBMGAN.png}
        \end{center}
    \end{frame}
    \section{Score-based Generative Modelling through SDE}
    \label{sec:reverse-sde} 
    \begin{frame}{Score-based Generative Modelling through SDE}
        \begin{center}
            The core idea of the paper can be described via the following picture
            \includegraphics[width=0.7\textwidth]{pics/NoiseDenoise.png}
        \end{center}
    \end{frame}

    \begin{frame}{Score-based Generative Modelling through SDE}
        \begin{center}
            \includegraphics[width=0.7\textwidth]{pics/NoiseDenoise2.png}
        \end{center}
        The main problem is to fit the neural network $\mathbf{s}_{\boldsymbol{\theta}}(\mathbf{x}(t), t)$ such that 
        $$
        \mathbf{s}_{\boldsymbol{\theta}}(\mathbf{x}(t), t) \approx \nabla_{\mathbf{x}} \log p_t(\mathbf{x})
        $$
    \end{frame}

    \begin{frame}{Loss function}
        \begin{multline*}
            \boldsymbol{\theta}^*=\underset{\boldsymbol{\theta}}{\arg \min } \\ \mathbb{E}_t\left\{ 
            \lambda(t) \mathbb{E}_{\mathbf{x}(0)} \mathbb{E}_{\mathbf{x}(t) \mid \mathbf{x}(0)}\left[\left\|\mathbf{s}_{\boldsymbol{\theta}}(\mathbf{x}(t), t)-\nabla_{\mathbf{x}(t)} \log p_{0 t}(\mathbf{x}(t) \mid \mathbf{x}(0))\right\|_2^2\right]\right\} .
        \end{multline*}
        
        where 
        $$
        \lambda \propto 1 / \mathbb{E}\left[\left\|\nabla_{\mathbf{x}(t)} \log p_{0 t}(\mathbf{x}(t) \mid \mathbf{x}(0))\right\|_2^2\right]
        $$
    \end{frame}

    \begin{frame}{2 approaches}
        In $\mathrm{d} \mathbf{x}=\mathbf{f}(\mathbf{x}, t) \mathrm{d} t+g(t) \mathrm{d} \mathbf{w}$ functions $\mathbf{f}$ and $g$ can be arbitrary, so we will consider 2 approaches
        \begin{itemize}
            \item Variance Exploding Approach
            \item Variance Preserving Approach
        \end{itemize}
    \end{frame}

    \begin{frame}{Variance Exploding}
        \begin{itemize}
            \item SDE 
        $$
        \mathrm{d} \mathbf{x}=\sqrt{\frac{\mathrm{d}\left[\sigma^2(t)\right]}{\mathrm{d} t}} \mathrm{~d} \mathbf{w}
        $$
            \item Forward Sampling
            $$
            \mathbf{x}_i=\mathbf{x}_{i-1}+\sqrt{\sigma_i^2-\sigma_{i-1}^2} \mathbf{z}_{i-1}
            $$
        \end{itemize}
    \end{frame}


    \begin{frame}{Variance Preserving}
        \begin{itemize}
        \item SDE 
            $$
        \mathrm{d} \mathbf{x}=-\frac{1}{2} \beta(t) \mathbf{x} \mathrm{d} t+\sqrt{\beta(t)} \mathrm{d} \mathbf{w}
        $$
        \item Forward Sampling
        $$
        \mathbf{x}_i=\sqrt{1-\beta_i} \mathbf{x}_{i-1}+\sqrt{\beta_i} \mathbf{z}_{i-1}
        $$
        \end{itemize}
    \end{frame}

    \begin{frame}{Ancestral Sampling for Variance Preserving}
        $$
        \mathbf{x}_{i-1}=\frac{1}{\sqrt{1-\beta_i}}\left(\mathbf{x}_i+\beta_i \mathbf{s}_{\boldsymbol{\theta}^*}\left(\mathbf{x}_i, i\right)\right)+\sqrt{\beta_i} \mathbf{z}_i, \quad i=N, N-1, \cdots, 1
        $$
    \end{frame}

    \begin{frame}{Reverse Diffusion}
        Given a forward SDE
$$
\mathrm{d} \mathbf{x}=\mathbf{f}(\mathbf{x}, t) \mathrm{d} t+\mathbf{G}(t) \mathrm{d} \mathbf{w}
$$
and suppose the following iteration rule is a discretization of it:
$$
\mathbf{x}_{i+1}=\mathbf{x}_i+\mathbf{f}_i\left(\mathbf{x}_i\right)+\mathbf{G}_i \mathbf{z}_i, \quad i=0,1, \cdots, N-1
$$
Thus, one can propose to discretize the reverse-time SDE
$$
\mathrm{d} \mathbf{x}=\left[\mathbf{f}(\mathbf{x}, t)-\mathbf{G}(t) \mathbf{G}(t)^{\top} \nabla_{\mathbf{x}} \log p_t(\mathbf{x})\right] \mathrm{d} t+\mathbf{G}(t) \mathrm{d} \overline{\mathbf{w}}
$$
which gives the following iteration rule for $i \in\{0,1, \cdots, N-1\}$ :
$$
\mathbf{x}_i=\mathbf{x}_{i+1}-\mathbf{f}_{i+1}\left(\mathbf{x}_{i+1}\right)+\mathbf{G}_{i+1} \mathbf{G}_{i+1}^{\top} \mathbf{s}_{\boldsymbol{\theta}^*}\left(\mathbf{x}_{i+1}, i+1\right)+\mathbf{G}_{i+1} \mathbf{z}_{i+1},
$$
where our trained score-based model $\mathbf{s}_{\boldsymbol{\theta}} *\left(\mathbf{x}_i, i\right)$.
    \end{frame}

    \begin{frame}{Predictor-Corrector Sampling}
        \begin{center}
            \includegraphics[width=\textwidth]{pics/PC.png}
        \end{center}
    \end{frame}

    \begin{frame}{\hypertarget{frame:res-ve}{Results. FID on CIFAR10. VE}}            
        \begin{center}
            \includegraphics[width=0.7\textwidth]{pics/VE_table.png}
        \end{center}
        As we can see Predictor-Corrector sampling gives the best performance in terms of
        FID metrics
    \end{frame}


    \begin{frame}{\hypertarget{frame:res-vp}{Results. FID on CIFAR10. VP}}            
        \begin{center}
            \includegraphics[width=0.7\textwidth]{pics/VP_table.png}
        \end{center}
        As we can see Predictor-Corrector sampling gives the best performance in terms of
        FID metrics
    \end{frame}
    \section{Conclusion}
    \begin{frame}{Conclusion}
        This work is devoted to examining the power of SDE theory in Image Generation.
        The work consists of 2 parts
        \begin{enumerate}
            \item The \ref{sec:energy-gan} part examines the paper [\ref{bib:energy-gan}] which main idea is based on the fact
            that GAN can be interpreted as Energy Based Model. Thus, one can easily
            apply Langevin Dynamics for it which can improve FID metrics for already pretrained 
            GAN model. The corresponding plot of FID behavior with dependence on  
            the Langevin steps is presented on \textcolor{blue}{\hyperlink{frame:res-energy}{this}} slide.
            \item The \ref{sec:reverse-sde} part reproduces experiments from the paper 
            [\ref{bib:reverse-sde}] that implements the following 
            idea. What if one can controllably transform the initial data disctibution into
            noise and reverse the process? Well, that can be done via reversing the SDE formula. Namely, 
            if one consider the SDE process that is defined by SDE 
            $\mathrm{d} \mathbf{x}=\mathbf{f}(\mathbf{x}, t) \mathrm{d} t+g(t) \mathrm{d} \mathbf{w}$
            then one can reverse it via 
            $\mathrm{d} \mathbf{x}=\left[\mathbf{f}(\mathbf{x}, t)-g(t)^2 \nabla_{\mathbf{x}} \log p_t(\mathbf{x})\right] \mathrm{d} t+g(t) \mathrm{d} \overline{\mathbf{w}}$. 
            In the paper several sampling technics are examined, and the results are finalized on 
            \textcolor{blue}{\hyperlink{frame:res-ve}{this}} and 
            \textcolor{blue}{\hyperlink{frame:res-vp}{this}} slides. 
        \end{enumerate}
    \end{frame}
    \section{References}

    \begin{frame}{References}
        \begin{enumerate}
            \item\label{bib:energy-gan} Tong Che, Ruixiang Zhang, Jascha Sohl-Dickstein, Hugo Larochelle, Liam Paull, Yuan Cao, Yoshua Bengio. \textit{Your GAN is Secretly an Energy-based Model and You Should use Discriminator Driven Latent Sampling}  \url{https://arxiv.org/abs/2003.06060}
            \item\label{bib:reverse-sde} Yang Song, Jascha Sohl-Dickstein, Diederik P. Kingma, Abhishek Kumar, Stefano Ermon, Ben Poole. \textit{Score-Based Generative Modeling through Stochastic Differential Equations} \url{https://arxiv.org/abs/2011.13456}            
            \item Github for DCGAN on CIFAR10 \url{https://github.com/csinva/gan-vae-pretrained-pytorch}
        \end{enumerate}
    \end{frame}

    % \section{Introduction}

In this work we consider noisy modification of the continiuos time HK model with leadership from \cite{leader_hk}.  
    % \section{Model}

In this section we state the noisy modification of the HK model from \cite{leader_hk}. 
The text is almost fully copypasted from the original paper, so it reasonable to make some 
antiplagiat fixes when the need arises.

The system of $N$ interacting agents with one leader with opinions in $\mathbb{R}^d$ is given by
\begin{equation}\label{eq:model_init} 
\left\{\begin{array}{l}
dx_0(t)=u(t)dt \\
dx_i(t)=\left[\sum_{j=1}^N a_{i j}\left(x_j(t)-x_i(t)\right)+c_i\left(x_0(t)-x_i(t)\right)\right] dt + \sigma_i(\mathbf{x}(t) - x_0(t)) dW^i_t,  \text { for } i=1, \ldots, N
\end{array}\right.
\end{equation}
with initial positions $x_j(0)$ such that $\Prob(x_j(0) = x_j^0) = 1$, where $x_j^0 = \operatorname*{const} \in \Real$ for $j=0,1, \ldots, N$. The state of the system, 
representing the agents' and the leader's opinions is 
$\mathbf{x}=\left(x_0, x_1, \ldots, x_N\right) \in \mathbb{R}^{(N+1)}$. We denote the leader by the index 0 and with the index $i=1, \ldots, N$ we denote the $N$ agents. 
The control function, which is a measurable function $t \mapsto u(t) \in \mathbb{R}$ satisfying the constraint $|u| \leq M$, acts only on the leader.
The first term in the dynamics $\sum_{j=1}^N a_{i j}\left(x_j-x_i\right)$, comes from the original HK model. The key idea of the $\mathrm{HK}$ model 
is that each agent updates his opinion by averaging the opinions of the neighbors with a rescaling factor
$$
a_{i j}=a\left(|x_i(t)-x_j(t)|\right)
$$
given by a function $a(r):[0, \infty) \rightarrow[0,1]$ of the distance $r$ between the opinions and representing the interaction rate dependence on the limited confidence domain. 
The function $a=a(r)$ is the following smooth cut-off function
$$
a(r)=a(r ; \delta, \varepsilon) \quad:= \begin{cases}1, & 0 \leq r \leq \delta, \\ \varphi(r), & \delta<r<(\delta+\varepsilon), \\ 0, & (\delta+\varepsilon) \leq r\end{cases}
$$
where $\delta$ is the bounded confidence, $\varphi(r)$ is a decreasing smooth function on $(\delta, \delta+\varepsilon]$, and $\varepsilon>0$ is a parameter of the HK model 
defining the width of the region where the cut-off function decays to zero.
The second term in the dynamics (1) models the action of the leader on the $i$ th agent. A leader can be defined as one agent with a high level of confidence and self-esteem, 
that has the ability to withstand criticism, so that its dynamics is not influenced by the other agents' opinions. The influence of the opinion of the leader on the group opinion 
in decision making is given by the term $c_i\left(x_0(t)-x_i(t)\right)$. The parameter
$$
c_i=\gamma \phi\left(|x_i-x_0|\right)
$$
represents the rate of relationship between the leader and the other agents, 
where $\phi:[0, \infty) \rightarrow(0,1]$ is a smooth non-increasing positive function 
such that $\phi(0)=1$ and $\lim _{r \rightarrow \infty} \phi(r)=\nu > 0$, and where the strength of the opinion 
leader is represented by the parameter $\gamma>0$. In other words the leader has the ability to influence every agent with a factor that is inversely 
proportional to its distance from the agent. The noise part
\[
    \sigma_i(\mathbf{x}(t) - x_0(t)) dW^i_t, 
\]
where $\{W^i_t\}_{i=1}^N$ are independent Weiner processes,  will be considered for different cases
\begin{enumerate}
    \item $\sigma_i(\mathbf{x}(t) - x_0(t)) = \sigma^2$
    \item $\sigma_i(\mathbf{x}(t) - x_0(t)) = \sigma^2(x_i(t) - x_0(t))$
    \item $\sigma_i(\mathbf{x}(t) - x_0(t)) = \sigma^2 h(x_i(t) - x_0(t))$, where $h$ is continuously differentiable function such that 
    \[
        h(0) = 0, \quad |h(\cdot)| \leq 1
    \]
\end{enumerate}
    % \section{Basic concepts and preliminaries} 

Again, copypaste

\textit{
The uncontrolled dynamics of system (1) is governed by local interactions and, in large time, leads to the
formation of clusters. Although, from the mathematical point of view, clusters are a stable configuration for
the system, we focus on the possibility to steer, using the leader’s action on the group, all agents to a same
unique opinion, that is, to have emergence of “consensus”
}

\begin{definition}
We call consensus a configuration in which the states of all agents are equal, i.e,
$$
\mathbf{x}^*=\left(x_0, x_1, \ldots, x_N\right) \in \mathbb{R}^{(N+1)} \quad \text { such that } \quad x_0=x_1=\cdots=x_N .
$$
\end{definition}

Note that it will be convinient for us to reformulate \eqref{eq:model_init} in terms of $y_i(t) := x_i(t) - x_0(t)$ for $i = 1, \ldots, N$:
\begin{equation}\label{eq:model}
    dy_i(t) = \left[-u(t) + \sum_{j=1}^N a_{i j}\left(y_j(t)-y_i(t)\right)-c_i y_i(t)\right] dt + \sigma_i(\Boldy(t)) dW^i_t, \qquad  i = 1, \ldots N,
\end{equation}
where $\Prob(y_j(0) = y_j^0) = 1$ for $j = 1, \ldots, N$ and $\Boldy_0 = (y_1^0, \ldots. y_N^0) \in \Real^N$. 
Thus, the consensus is the configuration $\Boldy^* = (y_1, \ldots, y_n) \in \Real^N$, where $y_1 = \ldots = y_N  = 0$. We will also need
\begin{definition}[modification from \cite{Khasminskii}]
    We will call the solution of the system \eqref{eq:model} exponentially $p$-stable if there exist positive constants $A$ and $\beta$ such that
    \[
       \Exp \|\Boldy(t)\|^p \leq A  \|\Boldy_0\|^p \exp\left\{-\beta t\right\}
    \]
    for all $\Boldy_0 = (y_1^0, \ldots. y_N^0) \in \Real^N$.
\end{definition}
and 
\begin{lemma}[Gronwall-Bellman Lemma \cite{Khasminskii}]\label{lemma:gronwall-bellman}
Let $f(t)$ and $g(t)$ be real-valued functions on $\Real_{\geq0}$ and let $f(t)$ be differentiable on $\Real_{>0}$. If
\[ 
f'(t) \leq f(t) g(t) 
\]
Then for $t \geq 0$
$$
f(t) \leq f(0) \exp \left\{\int_0^t g\left(\tau\right) d \tau\right\} .
$$
\end{lemma}


    % \section{Global stabilization}

\begin{theorem}\label{thm:global-stabilization}
    For all $M \in \Real_{>0}$ there exists smooth control $u(t)$ such that $|u(t)| < M$ for all $t \in \Real_{\geq 0 }$ and 
    \[
        \Exp \|\Boldy(t)\|^2 - \Exp \|\Boldy(s)\|^2 \leq \int_s^t  - \frac{\gamma \nu }{2} \Exp \|\Boldy(\tau)\|^2   + \sum_{i=1}^N \Exp \left[ \frac{\sigma_i(\Boldy(\tau))^2}{2} \right] d\tau 
    \]
\end{theorem}

\begin{proof}
    We consider the control function\footnote{The control function in \cite{leader_hk} has a similar form}
    \[
        u(t) = \alpha(t) \sum_{j=1}^N c_j y_j(t)
    \]
    where $\alpha(t)$ is smooth function of $y_1(t), \ldots, y_N(t)$ such that 
    \[
        \frac{1}{4} \min \left\{\frac{\gamma\phi\left( \max_{i}\left|y_i(t)\right|\right)}{N  \sum_{j=1}^N c_j}, \frac{2 M}{\gamma \sum_i\left|y_i(t)\right|}\right\} \leq \alpha(t) \leq \frac{1}{2} \min \left\{\frac{\gamma\phi\left( \max_{i}\left|y_i(t)\right|\right)}{N  \sum_{j=1}^N c_j}, \frac{2 M}{\gamma \sum_i\left|y_i(t)\right|}\right\} .
    \]
    Note that $u(t)$ is indeed bounded
    \[
        |u(t)| \leq  \alpha(t) \sum_{j=1}^N c_j |y_j(t)| \leq \alpha(t)  \sum_{j=1}^N \gamma |y_j(t)| \leq M
    \]
    Let us apply the Ito's lemma for $\|\Boldy(t)\|^2 = y_1^2(t) + \ldots + y_N^2(t)$:
    \begin{multline*}
        \|\Boldy(t)\|^2 -  \|\Boldy(s)\|^2 = \\ \sum_{i=1}^N\int_s^t \left[ \left(-u(\tau) + \sum_{j=1}^N a_{i j}\left(y_j(\tau)-y_i(\tau)\right)-c_i y_i(\tau)\right) y_i(\tau) + 
        \frac{\sigma_i(\Boldy(t))^2}{2} \right]d\tau + \\ 
        \sum_{i=1}^N  \int_s^{t}\sigma_i(\Boldy(\tau)) y_i(\tau) dW_{\tau}^i
    \end{multline*}
    The expected value of it
    \begin{multline}\label{eq:ExpIto}
        \Exp \|\Boldy(t)\|^2 -  \Exp \|\Boldy(s)\|^2 =  \sum_{i=1}^N \Exp \left[\int_s^t \left(-u(\tau) + \sum_{j=1}^N a_{i j}\left(y_j(\tau)-y_i(\tau)\right)-c_i y_i(\tau)\right) y_i(\tau) d\tau \right] + \\
        \sum_{i=1}^N \Exp \left[ \int_s^{t} \frac{\sigma_i(\Boldy(\tau))^2}{2} d\tau\right] + 0
    \end{multline}
    Let us consider separately
    \[
        \sum_{i=1}^N \Exp \left[\int_s^t \left(-u(\tau) + \sum_{j=1}^N a_{i j}\left(y_j(\tau)-y_i(\tau)\right)-c_i y_i(\tau)\right) y_i(\tau) d\tau \right]
    \]
    We need to rearrange terms in the following way
    \begin{equation}
        \Exp \left[\int_s^t - u(\tau) \sum_{i=1}^N y_i(\tau) + \sum_{i=1}^N \sum_{j=1}^N a_{i j}\left(y_j(\tau)-y_i(\tau)\right)y_i(\tau)- \sum_{i=1}^N c_i y_i(\tau)^2 d\tau \right]
    \end{equation}
    Let us consider the term
    \[
        \sum_{i=1}^N \sum_{j=1}^N a_{i j}\left(y_j(\tau)-y_i(\tau)\right)y_i(\tau)
    \]
    For $i = j$ the term $a_{i j}\left(y_j(\tau)-y_i(\tau)\right)y_i(t) = 0$. For all $i \neq j$ let us cluster the terms in the following pairs:
    \[
         a_{i j}\left(y_j(\tau)-y_i(\tau)\right)y_i(\tau) + a_{j i}\left(y_i(\tau)-y_j(\tau)\right)y_j(\tau) 
    \]
    Note that $a_{ij} = a_{ji}$ by definition of $a_{ij}$, so 
    \begin{multline*}
        a_{i j}\left(y_j(\tau)-y_i(\tau)\right)y_i(\tau) + a_{j i}\left(y_i(\tau)-y_j(t)\right)y_j(\tau)  = \\ 
         a_{i j}\left(y_j(\tau)-y_i(\tau)\right)y_i(\tau) + a_{i j }\left(y_i(\tau)-y_j(\tau)\right)y_j(\tau) = 
         - a_{i j} (y_i(\tau) - y_j(\tau)) ^ 2 \leq 0
    \end{multline*}
    Thus, \eqref{eq:ExpIto} becomes less than 
    \begin{multline*}
        \sum_{i=1}^N \Exp \left[ \int_s^{t} \frac{\sigma_i(\Boldy(\tau))^2}{2} d\tau\right]  +  \Exp \left[\int_s^t - u(\tau) \sum_{i=1}^N y_i(\tau) - \sum_{i=1}^N c_i y_i(\tau)^2 d\tau \right] =  \\ 
        \sum_{i=1}^N \Exp \left[ \int_s^{t} \frac{\sigma_i(\Boldy(\tau))^2}{2} d\tau\right]  +  \Exp \left[\int_s^t - \alpha(\tau) \sum_{i=1}^N \sum_{j=1}^N c_j y_j(\tau) y_i(\tau) - \sum_{i=1}^N c_i y_i(\tau)^2 d\tau \right] = 
    \end{multline*}
    Taking into account that, firstly,
    \begin{multline*}
        -\alpha(\tau) \sum_{i=1}^N \sum_{j=1}^N c_j y_j(\tau) y_i(\tau) \leq \alpha(\tau) \sum_{j=1}^N  c_j N \max_k |y_k(\tau)|^2  \leq  
        \|\Boldy(\tau)\|^2 \alpha(\tau) N \sum_{j=1}^N c_j \leq \\  \|\Boldy(\tau)\|^2  \frac{\gamma \phi(\max_k|y_k(\tau)|)}{2}
    \end{multline*}
    for the choice of $\alpha(\tau)$ and secondly, 
    \[
        c_j = \gamma \phi(|y_j(\tau)|) \geq \gamma \phi(\max_k|y_k(\tau)|) \Longrightarrow \sum_{i=1}^N c_i y_i(\tau)^2 \geq  \|\Boldy(\tau)\|^2  \gamma \phi(\max_k|y_k(\tau)|) 
    \]
    we deduce the following upper bound for  $\Exp \|\Boldy(t)\|^2 - \Exp \|\Boldy(s)\|^2$
    \begin{multline*}
        \Exp \|\Boldy(t)\|^2 - \Exp \|\Boldy(s)\|^2 \leq - \Exp\left[\int_s^t\frac{\gamma \phi(\max_k|y_k(\tau)|)}{2} \|\Boldy(\tau)\|^2 d\tau\right]+ \sum_{i=1}^N \Exp \left[ \int_s^{t} \frac{\sigma_i(\Boldy(\tau))^2}{2} d\tau\right]  \leq \\ 
        - \frac{\gamma \nu }{2} \int_s^t \Exp \|\Boldy(\tau)\|^2 d\tau + \sum_{i=1}^N \Exp \left[ \int_s^{t} \frac{\sigma_i(\Boldy(\tau))^2}{2} d\tau\right] 
    \end{multline*}
    since $\phi(\cdot) \geq \nu > 0$ by definition.
\end{proof}

\begin{corollary}\label{col:derivative}
    \[
        \left[ \Exp \|\Boldy(t)\|^2\right] ' \leq  - \frac{\gamma \nu }{2} \Exp \|\Boldy(t)\|^2   + \sum_{i=1}^N \Exp \left[ \frac{\sigma_i(\Boldy(t))^2}{2} \right]
    \] 
\end{corollary}
\begin{proof}
    Let us put
    \[
        s := t, \quad t := t + \Delta t
    \]
    where $\Delta t > 0$. Thus, we obtain the corollary statement via dividing both sides 
    of inequality from theorem~\ref{thm:global-stabilization} on $\Delta t$ and making $\Delta t \rightarrow 0+$.
    Since the control is smooth the function
    \[
        \Exp \|\Boldy(t)\| ^ 2 
    \]
    is differentiable, which finishes the proof.
\end{proof}
\subsection{Constant noise $\sigma_i(\Boldy(t)) = \sigma = \operatorname*{const}$.}\label{subsec:constant}
According to theorem \ref{thm:global-stabilization} we deduce that
\begin{multline*}
    \Exp \|\Boldy(t)\|^2  - \Exp \|\Boldy(s)\|^2 \leq  \int_s^t N \frac{\sigma^2}{2} - \frac{\gamma \nu }{2} \Exp \|\Boldy(\tau)\|^2 d\tau \Longleftrightarrow \\
    \left(\Exp \|\Boldy(t)\|^2 - N \frac{\sigma^2}{\gamma \nu}\right) - \left(\Exp \|\Boldy(s)\|^2 - N \frac{\sigma^2}{\gamma \nu }\right)  \leq 
     - \frac{\gamma \nu }{2} \int_s^t  \left[\Exp \|\Boldy(\tau)\|^2 - N \frac{\sigma^2}{\gamma \nu } \right] d\tau
\end{multline*}
So if we make the similar steps as in corollary \ref{col:derivative} and apply Gronwall-Bellman Lemma \ref{lemma:gronwall-bellman} to $$ \Exp \|\Boldy(t)\|^2 - N \frac{\sigma^2}{\gamma \nu}$$ we derive 
the following upper bound for $ \Exp \|\Boldy(t)\|^2$: 
\[
    \Exp \|\Boldy(t)\|^2 \leq N \frac{\sigma^2}{\gamma \nu} +  \left(\|\Boldy_0\|^2 - N \frac{\sigma^2}{\gamma \nu }\right) \exp \left\{-\frac{\gamma \nu }{2} t\right\} 
\]
which provide us the limit behaviour for $\Exp \|\Boldy(t)\|^2$. Indeed, the derived upper bound tends to $N \frac{\sigma^2}{\gamma \nu}$ as $t \rightarrow \infty$.

\subsection{Proportional noise $\sigma_i(\Boldy(t)) = \sigma y_i(t)$}\label{subsec:proportional}
According to theorem \ref{thm:global-stabilization} we deduce that
\begin{equation*}
    \Exp \|\Boldy(t)\|^2 -  \Exp\|\Boldy(s)\|^2 \leq   \int_s^t \frac{\sigma^2 - \gamma \nu }{2} \Exp \|\Boldy(\tau)\|^2 d\tau
\end{equation*}
So by applying corollary \ref{col:derivative} and Gronwall-Bellman Lemma \ref{lemma:gronwall-bellman} to $ \Exp \|\Boldy(t)\|^2 $, we derive
\[
    \Exp \|\Boldy(t)\|^2 \leq  \|\Boldy_0\|^2 \exp\left\{\frac{\sigma^2 - \gamma \nu }{2} t \right\}.
\]
In other words, the solution of \eqref{eq:model} is exponentially $p$-stable for $p=2$ if \[\sigma^2 < \gamma \nu\]
\subsection{Bounded noise $\sigma_i(\Boldy(t)) = \sigma h(y_i(t))$}
We assume that $h$ is smooth increasing function with $h(0) = 0$ and $|h(\cdot)| \leq 1$. Thus, there exists $K > 0$ such that
\[
    h(y) \leq K |y|, \quad y \in \Real.
\]
Then, on one hand
\[
    \sum_{i=1}^N \Exp \left[ \int_s^{t} \frac{\sigma_i(\Boldy(\tau))^2}{2} d\tau\right] \leq \frac{\sigma^2 K^2}{2}\int_s^t \Exp \|\Boldy(\tau)\|^2 d \tau 
\]
which provides us (according to subsection \ref{subsec:proportional})
\[
    \Exp \|\Boldy(t)\|^2 \leq  \|\Boldy_0\|^2 \exp\left\{\frac{K^2\sigma^2 - \gamma \nu }{2} t \right\}.
\]
And on the other hand
\begin{equation*}
    \Exp \|\Boldy(t)\|^2  - \Exp \|\Boldy(s)\|^2 \leq  \int_s^t N \frac{\sigma^2}{2} - \frac{\gamma \nu }{2} \Exp \|\Boldy(\tau)\|^2 d\tau
\end{equation*}
which provides us (according to subsection \ref{subsec:constant})
\[
    \Exp \|\Boldy(t)\|^2 \leq N \frac{\sigma^2}{\gamma \nu} +  \left(\|\Boldy_0\|^2 - N \frac{\sigma^2}{\gamma \nu }\right) \exp \left\{-\frac{\gamma \nu }{2} t\right\} 
\]
Thus,
\[
    \Exp \|\Boldy(t)\|^2 \leq \min\left(N \frac{\sigma^2}{\gamma \nu} +  
    \left(\|\Boldy_0\|^2 - N \frac{\sigma^2}{\gamma \nu }\right) \exp \left\{-\frac{\gamma \nu }{2} t\right\},  
    \|\Boldy_0\|^2 \exp\left\{\frac{K^2\sigma^2 - \gamma \nu }{2} t \right\} \right)
\]
Thus, we have exponential stability if $K^2\sigma^2 < \gamma\nu$. Moreover, for small times the bound
\[
    \Exp \|\Boldy(t)\|^2 \leq N \frac{\sigma^2}{\gamma \nu} +  \left(\|\Boldy_0\|^2 - N \frac{\sigma^2}{\gamma \nu }\right) \exp \left\{-\frac{\gamma \nu }{2} t\right\}
\]
guarantees us faster stabilization.

    
    % \bibliography{references}
    % \bibliographystyle{ieeetr}
\end{document}