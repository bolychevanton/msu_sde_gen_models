\section{Model}

In this section we state the noisy modification of the HK model from \cite{leader_hk}. 
The text is almost fully copypasted from the original paper, so it reasonable to make some 
antiplagiat fixes when the need arises.

The system of $N$ interacting agents with one leader with opinions in $\mathbb{R}^d$ is given by
\begin{equation}\label{eq:model_init} 
\left\{\begin{array}{l}
dx_0(t)=u(t)dt \\
dx_i(t)=\left[\sum_{j=1}^N a_{i j}\left(x_j(t)-x_i(t)\right)+c_i\left(x_0(t)-x_i(t)\right)\right] dt + \sigma_i(\mathbf{x}(t) - x_0(t)) dW^i_t,  \text { for } i=1, \ldots, N
\end{array}\right.
\end{equation}
with initial positions $x_j(0)$ such that $\Prob(x_j(0) = x_j^0) = 1$, where $x_j^0 = \operatorname*{const} \in \Real$ for $j=0,1, \ldots, N$. The state of the system, 
representing the agents' and the leader's opinions is 
$\mathbf{x}=\left(x_0, x_1, \ldots, x_N\right) \in \mathbb{R}^{(N+1)}$. We denote the leader by the index 0 and with the index $i=1, \ldots, N$ we denote the $N$ agents. 
The control function, which is a measurable function $t \mapsto u(t) \in \mathbb{R}$ satisfying the constraint $|u| \leq M$, acts only on the leader.
The first term in the dynamics $\sum_{j=1}^N a_{i j}\left(x_j-x_i\right)$, comes from the original HK model. The key idea of the $\mathrm{HK}$ model 
is that each agent updates his opinion by averaging the opinions of the neighbors with a rescaling factor
$$
a_{i j}=a\left(|x_i(t)-x_j(t)|\right)
$$
given by a function $a(r):[0, \infty) \rightarrow[0,1]$ of the distance $r$ between the opinions and representing the interaction rate dependence on the limited confidence domain. 
The function $a=a(r)$ is the following smooth cut-off function
$$
a(r)=a(r ; \delta, \varepsilon) \quad:= \begin{cases}1, & 0 \leq r \leq \delta, \\ \varphi(r), & \delta<r<(\delta+\varepsilon), \\ 0, & (\delta+\varepsilon) \leq r\end{cases}
$$
where $\delta$ is the bounded confidence, $\varphi(r)$ is a decreasing smooth function on $(\delta, \delta+\varepsilon]$, and $\varepsilon>0$ is a parameter of the HK model 
defining the width of the region where the cut-off function decays to zero.
The second term in the dynamics (1) models the action of the leader on the $i$ th agent. A leader can be defined as one agent with a high level of confidence and self-esteem, 
that has the ability to withstand criticism, so that its dynamics is not influenced by the other agents' opinions. The influence of the opinion of the leader on the group opinion 
in decision making is given by the term $c_i\left(x_0(t)-x_i(t)\right)$. The parameter
$$
c_i=\gamma \phi\left(|x_i-x_0|\right)
$$
represents the rate of relationship between the leader and the other agents, 
where $\phi:[0, \infty) \rightarrow(0,1]$ is a smooth non-increasing positive function 
such that $\phi(0)=1$ and $\lim _{r \rightarrow \infty} \phi(r)=\nu > 0$, and where the strength of the opinion 
leader is represented by the parameter $\gamma>0$. In other words the leader has the ability to influence every agent with a factor that is inversely 
proportional to its distance from the agent. The noise part
\[
    \sigma_i(\mathbf{x}(t) - x_0(t)) dW^i_t, 
\]
where $\{W^i_t\}_{i=1}^N$ are independent Weiner processes,  will be considered for different cases
\begin{enumerate}
    \item $\sigma_i(\mathbf{x}(t) - x_0(t)) = \sigma^2$
    \item $\sigma_i(\mathbf{x}(t) - x_0(t)) = \sigma^2(x_i(t) - x_0(t))$
    \item $\sigma_i(\mathbf{x}(t) - x_0(t)) = \sigma^2 h(x_i(t) - x_0(t))$, where $h$ is continuously differentiable function such that 
    \[
        h(0) = 0, \quad |h(\cdot)| \leq 1
    \]
\end{enumerate}