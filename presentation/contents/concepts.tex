\section{Basic concepts and preliminaries} 

Again, copypaste

\textit{
The uncontrolled dynamics of system (1) is governed by local interactions and, in large time, leads to the
formation of clusters. Although, from the mathematical point of view, clusters are a stable configuration for
the system, we focus on the possibility to steer, using the leader’s action on the group, all agents to a same
unique opinion, that is, to have emergence of “consensus”
}

\begin{definition}
We call consensus a configuration in which the states of all agents are equal, i.e,
$$
\mathbf{x}^*=\left(x_0, x_1, \ldots, x_N\right) \in \mathbb{R}^{(N+1)} \quad \text { such that } \quad x_0=x_1=\cdots=x_N .
$$
\end{definition}

Note that it will be convinient for us to reformulate \eqref{eq:model_init} in terms of $y_i(t) := x_i(t) - x_0(t)$ for $i = 1, \ldots, N$:
\begin{equation}\label{eq:model}
    dy_i(t) = \left[-u(t) + \sum_{j=1}^N a_{i j}\left(y_j(t)-y_i(t)\right)-c_i y_i(t)\right] dt + \sigma_i(\Boldy(t)) dW^i_t, \qquad  i = 1, \ldots N,
\end{equation}
where $\Prob(y_j(0) = y_j^0) = 1$ for $j = 1, \ldots, N$ and $\Boldy_0 = (y_1^0, \ldots. y_N^0) \in \Real^N$. 
Thus, the consensus is the configuration $\Boldy^* = (y_1, \ldots, y_n) \in \Real^N$, where $y_1 = \ldots = y_N  = 0$. We will also need
\begin{definition}[modification from \cite{Khasminskii}]
    We will call the solution of the system \eqref{eq:model} exponentially $p$-stable if there exist positive constants $A$ and $\beta$ such that
    \[
       \Exp \|\Boldy(t)\|^p \leq A  \|\Boldy_0\|^p \exp\left\{-\beta t\right\}
    \]
    for all $\Boldy_0 = (y_1^0, \ldots. y_N^0) \in \Real^N$.
\end{definition}
and 
\begin{lemma}[Gronwall-Bellman Lemma \cite{Khasminskii}]\label{lemma:gronwall-bellman}
Let $f(t)$ and $g(t)$ be real-valued functions on $\Real_{\geq0}$ and let $f(t)$ be differentiable on $\Real_{>0}$. If
\[ 
f'(t) \leq f(t) g(t) 
\]
Then for $t \geq 0$
$$
f(t) \leq f(0) \exp \left\{\int_0^t g\left(\tau\right) d \tau\right\} .
$$
\end{lemma}

